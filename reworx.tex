\documentclass[11pt,a4paper,twoside]{article}

% LaTeX-Umsetzung der "Richtlinien f�r Projekt- und Diplomarbeiten"
% der LFE Medieninformatik, LMU M�nchen. (Autor: Richard Atterer, 27.9.2006, 23.10.2007), Bug-Fixing Mark Kaczkowski (23.6.2008)

\usepackage[T1]{fontenc} % sonst geht \hyphenation nicht mit Umlauten
\usepackage[latin1]{inputenc} % man kann schreiben ����, statt "a"o"u"s
%\usepackage[utf8]{inputenc} % wie oben, aber UTF-8 als Encoding statt ISO-8859-1 (latin1)
\usepackage[ngerman,english]{babel} % deutsche Trennregeln, "Inhaltsverzeichnis" etc.
%\usepackage{ngerman} % Alternative zum Babel-Paket oben
\usepackage{mathptmx} % Times-Roman-Schrift (auch f�r mathematische Formeln)

% fonts
\fontfamily{ptm}\selectfont

% Zum Setzen von URLs
\usepackage{color}
\definecolor{darkred}{rgb}{.25,0,0}
\definecolor{darkgreen}{rgb}{0,.2,0}
\definecolor{darkmagenta}{rgb}{.2,0,.2}
\definecolor{darkcyan}{rgb}{0,.15,.15}
\usepackage[plainpages=false,bookmarks=true,bookmarksopen=true,colorlinks=true,
  linkcolor=darkred,citecolor=darkgreen,filecolor=darkmagenta,
  menucolor=darkred,urlcolor=darkcyan]{hyperref}

% pdflatex: Bilder in den Formaten .jpeg, .png und .pdf
% latex: Bilder im .eps-Format
\usepackage{graphicx}

\usepackage{fancyhdr} % Positionierung der Seitenzahlen
\fancyhead[LE,RO,LO,RE]{}
\fancyfoot[CE,CO,RE,LO]{}
\fancyfoot[LE,RO]{\Roman{page}}
\renewcommand{\headrulewidth}{0pt}
\setlength{\headheight}{13.6pt} % behebt headheight Warning

% Korrektes Format f�r Nummerierung von Abbildungen (figure) und
% Tabellen (table): <Kapitelnummer>.<Abbildungsnummer>
\makeatletter
\@addtoreset{figure}{section}
\renewcommand{\thefigure}{\thesection.\arabic{figure}}
\@addtoreset{table}{section}
\renewcommand{\thetable}{\thesection.\arabic{table}}
\makeatother

\sloppy % Damit LaTeX nicht so viel �ber "overfull hbox" u.�. meckert

% R�nder
\addtolength{\topmargin}{-16mm}
\setlength{\oddsidemargin}{25mm}
\setlength{\evensidemargin}{35mm}
\addtolength{\oddsidemargin}{-1in}
\addtolength{\evensidemargin}{-1in}
\setlength{\textwidth}{15cm}
\addtolength{\textheight}{34mm}
%______________________________________________________________________

\begin{document}
\selectlanguage{english}

\pagestyle{empty} % Vorerst keine Seitenzahlen
\pagenumbering{alph} % Unsichtbare alphabetische Nummerierung

\begin{center}

\vspace*{5cm}
{\Large Bachelor Thesis}
\vspace{.4cm}

{\LARGE TODO: TITLE THAT SAYS MORE THAN "TITLE"}
\vspace{1.8cm}

by
\bigskip

{\Large Felix Hamann}\\
\smallskip

Born on September 21st, 1989 in Grafing bei M�nchen\\
Matriculation number: 10809408
\vspace{4cm}

{\Large Ludwig-Maximilians-Universit�t M�nchen}
\smallskip

Department of Computer Science\\
Institute for Software and Multimedia Technology\\
Prof. Dr. Heinrich Hu�mann
\end{center}
\vfill

\begin{tabular}{ll}
Supervised by: & Mohammed Khamis\\
& Dr. Steffen Jost\\
Submitted on: & mmmm dd, yyyy
\end{tabular}
%______________________________________________________________________

\clearpage
\section*{Abstract}

rework uniworx...

\clearpage
\section*{Scope}

(Kopie der Original-Aufgabenstellung)

TODO: Jost fragen? Khamis fragen?

\clearpage
\section*{Confirmation}
I confirm that I independently prepared the thesis and that I used
only the references and auxiliary means indicated in the thesis.
\vspace{2cm}

\noindent\makebox[8cm]{\dotfill}
\vspace{.7cm}

\noindent Munich, \today

%______________________________________________________________________

\cleardoublepage
\pagestyle{fancy}
\pagenumbering{roman} % R�mische Seitenzahlen
\setcounter{page}{1}

% Inhaltsverzeichnis erzeugen
\tableofcontents

%Abbildungsverzeichnis erzeugen - normalerweise nicht n�tig
%\cleardoublepage
%\listoffigures
%______________________________________________________________________

\cleardoublepage

% Arabische Seitenzahlen
\pagenumbering{arabic}
\setcounter{page}{1}
% Ge�ndertes Format f�r Seitenr�nder, arabische Seitenzahlen
\fancyhead[LE,RO]{\rightmark}
\fancyhead[LO,RE]{\leftmark}
\fancyfoot[LE,RO]{\thepage}

\section{Introduction}

Die Ludwig-Maximilians-Universit�t M�nchen benutzt seit mehreren Jahren (Wie lange?) eine Online-Plattform namens Uniworx um die Abgaben, Klausuranmeldungen und �bungsgruppen der (Medien-/Wirtschaft-)Informatik-Studenten zu organisieren.

Die 1. Version von \textbf{UniWorX} entstand im Rahmen einer Bachelorarbeit (Stimmt? Quelle?) und erf�llte x Jahre ihren Zweck. Nach diesen x Jahren wurde sie in wenigen N�chten von MANFRED MUSTERMAX neu geschrieben, da die 1. Implementierung sich als nicht mehr wartbar erwies und obligatorische Zusatzfeatures nicht implementiert werden konnten.

Die zweite Version ist in PERL geschrieben und diente als gro�e Inspiration f�r die Neuimplementierung die in dieser Arbeit thematisiert werden soll.

Es soll einerseits auf die technischen Aspekte der neuen Version "ReWorX", als auch auf die HCI\footnote{Human-Computer-Interaktion}-Aspekte einer solchen Plattform. In \autoref{sec:implementation} wird n�her auf die Details der Implemntierung eingegangen.
\autoref{sec:userstudies} wird sich mit den Usuability-Studien befassen, die im Rahmen dieser Arbeit durchgef�hrt wurden. Die Aufgaben in dieser Arbeit waren sehr unterschiedlich:

\begin{enumerate}
  \item Es bedurfte einer gr�ndlichen Analyse des bestehenden Systems in Bezug auf User-Flows\footnote{Benutzerf�hrung auf der Website}
  \item Es musste festegestellt werden welche der bestehenden Seiten wirklich n�tig waren und welche wom�gloich abgek�rzt werden k�nnten (durch hover-Men�s, reine Anwendung gestatlertischer Merkmale wie Fitt'S Law).
\end{enumerate}


\cleardoublepage % Neue rechte Seite anfangen
\section{America First? Mobile First!}

90\% of the orders AliBaba\footnote{chinese online discount retail} received during their 2018 chinese-new-year promotion campaign where placed from mobile devices (CITE!).
This is still far from the internet usage on mobile-devices in central europe, where (according to NAMEDROPPING) around 60\% of all page views came from users on mobile devices (phones / tablets).

\subsection{Mobile Website vs. Responsive Design}

Zwei Teile blabla

\subsection{Implementierung} \label{sec:implementation}

\subsection{Benutzer-Studien} \label{sec:userstudies}

\begin{figure}%[btph]
  %% Datei ``beispielbild.eps'' oder ``beispielbild.png'', zentriert
  %\begin{center}\includegraphics{beispielbild}\end{center}

  %% Datei auf 8cm Breite verkleinert/vergr��ert
  %\includegraphics[width=8cm]{beispielbild}
  %% Datei auf ganze Breite des Texts vergr��ert
  %\includegraphics[width=\columnwidth]{beispielbild}
  %% Datei auf 60% der Textbreite verkleinert/vergr��ert
  %\includegraphics[width=.6\columnwidth]{beispielbild}
  %% Weitere Optionen (Ausschnitt, drehen etc.) in der Doku zum graphicx-Paket

  \begin{center}\LARGE [BILD]\end{center}
  \caption{Bildunterschrift}
  \label{fig:beispielbild}
\end{figure}

%\_____________________________________________________________________

\cleardoublepage
\section{Zusammenfassung}

\begin{figure}
  \begin{center}\LARGE [BILD]\end{center}
  \caption{Bild}
  \label{fig:beispielbild3}
\end{figure}
%______________________________________________________________________

\cleardoublepage
\fancyhead[LE,RO,LO,RE]{} % Keine Kopfzeile mehr oben auf jeder Seite
\section*{Inhalt der beigelegten CD}
%______________________________________________________________________

\cleardoublepage
\bibliographystyle{unsrt}
\bibliography{literature}

\end{document}
